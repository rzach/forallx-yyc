
\chapter{Normal forms}\setcounter{ProbPart}{0}

\problempart
\label{pr.DNF}
Consider the following sentences:
	\begin{earg}
		\item $(A \eif \enot B)$
		\item $\enot (A \eiff B)$
		\item $(\enot A \eor \enot (A \eand B))$
		\item $(\enot (A \eif B ) \eand (A \eif C))$
		\item $(\enot (A \eor B) \eiff ((\enot C \eand \enot A) \eif \enot B))$
		\item $((\enot (A \eand \enot B) \eif C) \eand \enot (A \eand D))$
	\end{earg}
For each sentence, find an equivalent sentence in DNF and one in CNF.
\myanswer{We give a solution for (2). The truth table for $\enot (A \eiff B)$ is:
\begin{center}
\begin{tabular}{c c | l}
$A$ & $B$ & $\enot (A \eiff B)$\\
\hline
T & T & F \\
T & F & T \\
F & T & T \\
F & F & F \\
\end{tabular}
\end{center}
A sentence in DNF can be read off from lines 2 and 3:
\[(A \eand \enot B) \eor (\enot A \eand B)\]
and one in CNF from lines 1 and~4:
\[(\enot A \eor \enot B) \eand (A \eor B).\]}

\stepcounter{chapter} % Functional completeness

\chapter{Proving equivalences}\setcounter{ProbPart}{0}

\problempart
\label{pr.DNF2}
Consider the following sentences:
\begin{earg}
	\item $(A \eif \enot B)$
	\item $\enot (A \eiff B)$
	\item $(\enot A \eor \enot (A \eand B))$
	\item $(\enot (A \eif B ) \eand (A \eif C))$
	\item $(\enot (A \eor B) \eiff ((\enot C \eand \enot A) \eif \enot B))$
	\item $((\enot (A \eand \enot B) \eif C) \eand \enot (A \eand D))$
\end{earg}
For each sentence, find an equivalent sentence in DNF and one inCNF by giving a chain of equivalences. Use (Id), (Absorp), and (Simp) to simplify your sentences as much as possible.
\myanswer{We give a solution for (2). Removing `$\eiff$' and pushing negations inward is common to both:
\begin{align*}
	& \enot (A \eiff B)\\
	& \enot((A \eif B) \eand (B \eif A)) && \text{Bicond}\\
	& \enot((\enot A \eor B) \eand (B \eif A)) && \text{Cond}\\
	& \enot((\enot A \eor B) \eand (\enot B \eor A)) && \text{Cond}\\
	& \enot(\enot A \eor B) \eor \enot(\enot B \eor A) && \text{DeM}\\
	& (\enot\enot A \eand \enot B) \eor (\enot\enot B \eand \enot A) && \text{DeM}\\
	& (A \eand \enot B) \eor (\enot\enot B \eand \enot A) && \text{DN}\\
	& (A \eand \enot B) \eor (B \eand \enot A) && \text{DN}
\intertext{The result is now in DNF. To obtain a CNF, we keep going, using (Comm) and (Dist):}
& ((A \eand \enot B) \eor B) \eand ((A \eand \enot B) \eor \enot A) && \text{Dist}\\
& (B \eor (A \eand \enot B)) \eand ((A \eand \enot B) \eor \enot A) && \text{Comm}\\
& ((B \eor A) \eand (B \eor \enot B)) \eand ((A \eand \enot B) \eor \enot A) && \text{Dist}\\
& ((B \eor A) \eand (B \eor \enot B)) \eand (\enot A \eor (A \eand \enot B)) && \text{Comm}\\
& ((B \eor A) \eand (B \eor \enot B)) \eand ((\enot A \eor A) \eand (\enot A \eor \enot B)) && \text{Dist}\\
\intertext{The result can be simplified using (Simp):}
& (B \eor A) \eand ((\enot A \eor A) \eand (\enot A \eor \enot B)) && \text{Simp}\\
& (B \eor A) \eand (\enot A \eor \enot B) && \text{Simp}
\end{align*}}

\stepcounter{chapter} % Soundness