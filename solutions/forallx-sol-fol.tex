%!TEX root = forallxyyc-solutions.tex
%\part{First-order logic}
%\label{ch.FOL}
%\addtocontents{toc}{\protect\mbox{}\protect\hrulefill\par}

\stepcounter{chapter} % Building blocks

\chapter{Sentences with one quantifier}\label{s:MoreMonadic}\setcounter{ProbPart}{0}
\problempart
\label{pr.BarbaraEtc}
Here are the syllogistic figures identified by Aristotle and his successors, along with their medieval names:
\begin{compactlist}
	\item \textbf{Barbara.} All G are F. All H are G. So:  All H are F
	\item[] \myanswer{$\forall x (\atom{G}{x} \eif \atom{F}{x}), \forall x (\atom{H}{x} \eif \atom{G}{x}) \therefore \forall x (\atom{H}{x} \eif \atom{F}{x})$}
	\item \textbf{Celarent.} No G are F. All H are G. So: No H are F
	\item[] \myanswer{$\forall x (\atom{G}{x} \eif \enot \atom{F}{x}), \forall x (\atom{H}{x} \eif \atom{G}{x}) \therefore \forall x (\atom{H}{x} \eif \enot \atom{F}{x})$}
	\item \textbf{Ferio.} No G are F. Some H is G. So: Some H is not F
	\item[] \myanswer{$\forall x (\atom{G}{x} \eif \enot \atom{F}{x}), \exists x (\atom{H}{x} \eand  \atom{G}{x}) \therefore \exists x (\atom{H}{x} \eand \enot \atom{F}{x})$}
	\item \textbf{Darii.} All G are H. Some H is G. So: Some H is F.
	\item[] \myanswer{$\forall x (\atom{G}{x} \eif \atom{F}{x}), \exists x (\atom{H}{x} \eand  \atom{G}{x}) \therefore \exists x (\atom{H}{x} \eand  \atom{F}{x})$}
	\item \textbf{Camestres.} All F are G. No H are G. So: No H are F.
	\item[] \myanswer{$\forall x (\atom{F}{x} \eif \atom{G}{x}), \forall x (\atom{H}{x} \eif \enot \atom{G}{x}) \therefore \forall x (\atom{H}{x} \eif \enot \atom{F}{x})$}
	\item \textbf{Cesare.} No F are G. All H are G. So: No H are F.
	\item[] \myanswer{$\forall x (\atom{F}{x} \eif \enot \atom{G}{x}), \forall x (\atom{H}{x} \eif \atom{G}{x}) \therefore \forall x (\atom{H}{x} \eif \enot \atom{F}{x})$}
	\item \textbf{Baroko.} All F are G. Some H is not G. So: Some H is not F.
	\item[] \myanswer{$\forall x (\atom{F}{x} \eif \atom{G}{x}), \exists x (\atom{H}{x} \eand \enot \atom{G}{x}) \therefore \exists x (\atom{H}{x} \eand \enot \atom{F}{x})$}
	\item \textbf{Festino.} No F are G. Some H are G. So: Some H is not F.
	\item[] \myanswer{$\forall x (\atom{F}{x} \eif \enot \atom{G}{x}), \exists x (\atom{H}{x} \eand \atom{G}{x}) \therefore \exists x (\atom{H}{x} \eand \enot \atom{F}{x})$}
	\item \textbf{Datisi.} All G are F. Some G is H. So: Some H is F.
	\item[] \myanswer{$\forall x (\atom{G}{x} \eif \atom{F}{x}), \exists x (\atom{G}{x} \eand \atom{H}{x}) \therefore \exists x (\atom{H}{x} \eand \atom{F}{x})$}
	\item \textbf{Disamis.} Some G is F. All G are H. So: Some H is F.
	\item[] \myanswer{$\exists x (\atom{G}{x} \eand \atom{F}{x}), \forall x (\atom{G}{x} \eif \atom{H}{x}) \therefore \exists x (\atom{H}{x} \eand \atom{F}{x})$}
	\item \textbf{Ferison.} No G are F. Some G is H. So: Some H is not F.
	\item[] \myanswer{$\forall x (\atom{G}{x} \eif \enot \atom{F}{x}), \exists x (\atom{G}{x} \eand \atom{H}{x}) \therefore \exists x (\atom{H}{x} \eand \enot \atom{F}{x})$}
	\item \textbf{Bokardo.} Some G is not F. All G are H. So:  Some H is not F.
	\item[] \myanswer{$\exists x (\atom{G}{x} \eand \enot \atom{F}{x}), \forall x (\atom{G}{x} \eif \atom{H}{x}) \therefore \exists x (\atom{H}{x} \eand \enot \atom{F}{x})$}
	\item \textbf{Camenes.} All F are G. No G are H So: No H is F.
	\item[] \myanswer{$\forall x (\atom{F}{x} \eif \atom{G}{x}), \forall x (\atom{G}{x} \eif \enot \atom{H}{x}) \therefore \forall x (\atom{H}{x} \eif \enot \atom{F}{x})$}
	\item \textbf{Dimaris.} Some F is G. All G are H. So: Some H is F.
	\item[] \myanswer{$\exists x (\atom{F}{x} \eand \atom{G}{x}), \forall x (\atom{G}{x} \eif \atom{H}{x}) \therefore \exists x (\atom{H}{x} \eand \atom{F}{x})$}
	\item \textbf{Fresison.} No F are G. Some G is H. So: Some H is not F.
	\item[] \myanswer{$\forall x (\atom{F}{x} \eif \enot \atom{G}{x}), \exists x (\atom{G}{x} \eand \atom{H}{x}) \therefore \exists (\atom{H}{x} \eand \enot \atom{F}{x})$}
\end{compactlist}
Symbolize each argument in FOL.

\
\problempart
\label{pr.FOLvegetarians}
Using the following symbolization key:
\begin{ekey}
\item[\text{domain}] people
\item[\atom{K}{x}] \gap{x} knows the combination to the safe
\item[\atom{S}{x}] \gap{x} is a spy
\item[\atom{V}{x}] \gap{x} is a vegetarian
%\item[\atom{T}{x,y}] \gap{x} trusts \gap{y}.
\item[h] Hofthor
\item[i] Ingmar
\end{ekey}
symbolize the following sentences in FOL:
\begin{compactlist}
\item Neither Hofthor nor Ingmar is a vegetarian.
\item[] \myanswer{$\enot \atom{V}{h} \eand \enot \atom{V}{i}$}
\item No spy knows the combination to the safe.
\item[] \myanswer{$\forall x (\atom{S}{x} \eif \enot \atom{K}{x})$}
\item No one knows the combination to the safe unless Ingmar does.
\item[] \myanswer{$\forall x \enot \atom{K}{x} \eor \atom{K}{i}$}
\item Hofthor is a spy, but no vegetarian is a spy.
\item[] \myanswer{$\atom{S}{h} \eand \forall x(\atom{V}{x} \eif \enot \atom{S}{x})$}
%\item Hofthor trusts a vegetarian.
%\item Everyone who trusts Ingmar trusts a vegetarian.
%\item Everyone who trusts Ingmar trusts someone who trusts a vegetarian.
%\item Only Ingmar knows the combination to the safe.
%\item Ingmar trusts Hofthor, but no one else.
%\item The person who knows the combination to the safe is a vegetarian.
%\item The person who knows the combination to the safe is not a spy.
\end{compactlist}

\solutions
\problempart\label{pr.FOLalligators}
Using this symbolization key:
\begin{ekey}
\item[\text{domain}] all animals
\item[\atom{A}{x}] \gap{x} is an alligator
\item[\atom{M}{x}] \gap{x} is a monkey
\item[\atom{R}{x}] \gap{x} is a reptile
\item[\atom{Z}{x}] \gap{x} lives at the zoo
\item[a] Amos
\item[b] Bouncer
\item[c] Cleo
\end{ekey}
symbolize each of the following sentences in FOL:
\begin{compactlist}
\item Amos, Bouncer, and Cleo all live at the zoo.
\item[] \myanswer{$\atom{Z}{a} \eand \atom{Z}{b} \eand \atom{Z}{c}$}
\item Bouncer is a reptile, but not an alligator.
\item[] \myanswer{$\atom{R}{b} \eand \enot \atom{A}{b}$}
%\item If Cleo loves Bouncer, then Bouncer is a monkey.
%\item If both Bouncer and Cleo are alligators, then Amos loves them both.
\item Some reptile lives at the zoo.
\item[] \myanswer{$\exists x (\atom{R}{x} \eand \atom{Z}{x})$}
\item Every alligator is a reptile.
\item[] \myanswer{$\forall x(\atom{A}{x} \eif \atom{R}{x})$}
\item Any animal that lives at the zoo is either a monkey or an alligator.
\item[] \myanswer{$\forall x(\atom{Z}{x} \eif (\atom{M}{x} \eor \atom{A}{x}))$}
\item There are reptiles which are not alligators.
\item[] \myanswer{$\exists x (\atom{R}{x} \eand \enot \atom{A}{x})$}
%\item Cleo loves a reptile.
%\item Bouncer loves all the monkeys that live at the zoo.
%\item All the monkeys that Amos loves love him back.
\item If any animal is an reptile, then Amos is.
\item[] \myanswer{$\exists x\, \atom{R}{x} \eif \atom{R}{a}$}
\item If any animal is an alligator, then it is a reptile.
\item[] \myanswer{$\forall x(\atom{A}{x} \eif \atom{R}{x})$}
%\item Every monkey that Cleo loves is also loved by Amos.
%\item There is a monkey that loves Bouncer, but sadly Bouncer does not reciprocate this love.
\end{compactlist}

\problempart
\label{pr.FOLarguments}
For each argument, write a symbolization key and symbolize the argument in FOL.
\begin{compactlist}
\item Willard is a logician. All logicians wear funny hats. So Willard wears a funny hat
\myanswer{
\begin{ekey}
\item[\text{domain}] people
\item[\atom{L}{x}] \gap{x} is a logician
\item[\atom{H}{x}] \gap{x} wears a funny hat
\item[i] Willard
\end{ekey}
$\atom{L}{i}, \forall x (\atom{L}{x} \eif \atom{H}{x}) \therefore \atom{H}{i}$}
\item Nothing on my desk escapes my attention. There is a computer on my desk. As such, there is a computer that does not escape my attention.
\myanswer{
\begin{ekey}
\item[\text{domain}] physical things
\item[\atom{D}{x}] \gap{x} is on my desk
\item[\atom{E}{x}] \gap{x} escapes my attention
\item[\atom{C}{x}] \gap{x} is a computer
\end{ekey}
$\forall x (\atom{D}{x} \eif \enot \atom{E}{x}), \exists x(\atom{D}{x} \eand \atom{C}{x}) \therefore \exists x (\atom{C}{x} \eand \enot \atom{E}{x})$}
\item All my dreams are black and white. Old TV shows are in black and white. Therefore, some of my dreams are old TV shows.
\myanswer{
\begin{ekey}
\item[\text{domain}] episodes (psychological and televised)
\item[\atom{D}{x}] \gap{x} is one of my dreams
\item[\atom{B}{x}] \gap{x} is in black and white
\item[\atom{O}{x}] \gap{x} is an old TV show
\end{ekey}
$\forall x (\atom{D}{x} \eif \atom{B}{x}), \forall x (\atom{O}{x} \eif \atom{B}{x}) \therefore \exists x (\atom{D}{x} \eand \atom{O}{x})$. \\Comment: generic statements are tricky to deal with. Does the second sentence mean that \emph{all} old TV shows are in black and white; or that most of them are; or that most of the things which are in black and white are old TV shows? I have gone with the former, but it is not clear that FOL deals with these well.}
\item Neither Holmes nor Watson has been to Australia. A person could see a kangaroo only if they had been to Australia or to a zoo. Although Watson has not seen a kangaroo, Holmes has. Therefore, Holmes has been to a zoo.
\myanswer{
\begin{ekey}
\item[\text{domain}] people
\item[\atom{A}{x}] \gap{x} has been to Australia
\item[\atom{K}{x}] \gap{x} has seen a kangaroo
\item[\atom{Z}{x}] \gap{x} has been to a zoo
\item[h] Holmes
\item[a] Watson
\end{ekey}
$\enot \atom{A}{h} \eand \enot \atom{A}{a}, \forall x(\atom{K}{x} \eif (\atom{A}{x} \eor \atom{Z}{x})), \enot \atom{K}{a} \eand \atom{K}{h} \therefore \atom{Z}{h}$}
\item No one expects the Spanish Inquisition. No one knows the troubles I've seen. Therefore, anyone who expects the Spanish Inquisition knows the troubles I've seen.
\myanswer{
\begin{ekey}
\item[\text{domain}] people
\item[\atom{S}{x}] \gap{x} expects the Spanish Inquisition
\item[\atom{T}{x}] \gap{x} knows the troubles I've seen
\end{ekey}
$\forall x\enot \atom{S}{x}, \forall x \enot \atom{T}{x} \therefore \forall x (\atom{S}{x} \eif \atom{T}{x})$}
\item All babies are illogical. Nobody who is illogical can manage a crocodile. Berthold is a baby. Therefore, Berthold is unable to manage a crocodile.
\myanswer{\begin{ekey}
\item[\text{domain}] people
\item[\atom{B}{x}] \gap{x} is a baby
\item[\atom{I}{x}] \gap{x} is illogical
\item[\atom{C}{x}] \gap{x} can manage a crocodile
\item[b] Berthold
\end{ekey}
$\forall x (\atom{B}{x} \eif \atom{I}{x}), \forall x (\atom{I}{x} \eif \enot \atom{C}{x}), \atom{B}{b} \therefore \enot \atom{C}{b}$}
\end{compactlist}

\chapter{Multiple generality}\setcounter{ProbPart}{0}
\problempart
Using this symbolization key:
\begin{ekey}
\item[\text{domain}] all animals
\item[\atom{A}{x}] \gap{x} is an alligator
\item[\atom{M}{x}] \gap{x} is a monkey
\item[\atom{R}{x}] \gap{x} is a reptile
\item[\atom{Z}{x}] \gap{x} lives at the zoo
\item[\atom{L}{x,y}] \gap{x} loves \gap{y}
\item[a] Amos
\item[b] Bouncer
\item[c] Cleo
\end{ekey}
symbolize each of the following sentences in FOL:
\begin{compactlist}
\item If Cleo loves Bouncer, then Bouncer is a monkey.
\item[] \myanswer{$\atom{L}{c,b} \eif \atom{M}{b}$}
\item If both Bouncer and Cleo are alligators, then Amos loves them both.
\item[] \myanswer{$(\atom{A}{b} \eand \atom{A}{c}) \eif (\atom{L}{a,b} \eand \atom{L}{a,c})$}
%\item Some reptile lives at the zoo.
%\item Every alligator is a reptile.
%\item Any animal that lives at the zoo is either a monkey or an alligator.
%\item There are reptiles which are not alligators.
\item Cleo loves a reptile.
\item[] \myanswer{$\exists x(\atom{R}{x} \eand \atom{L}{c,x})$\\Comment: this English expression is ambiguous; in some contexts, it can be read as a generic, along the lines of `Cleo loves reptiles'. (Compare `I do love a good pint'.) }
\item Bouncer loves all the monkeys that live at the zoo.
\item[] \myanswer{$\forall x ((\atom{M}{x} \eand \atom{Z}{x}) \eif \atom{L}{b,x})$}\item All the monkeys that Amos loves love him back.
\item[] \myanswer{$\forall x ((\atom{M}{x} \eand \atom{L}{a,x}) \eif \atom{L}{x,a})$}
%\item If any animal is an reptile, then Amos is.
%\item If any animal is an alligator, then it is a reptile.
\item Every monkey that Cleo loves is also loved by Amos.
\item[] \myanswer{$\forall x ((\atom{M}{x} \eand \atom{L}{c,x}) \eif \atom{L}{a,x})$}
\item There is a monkey that loves Bouncer, but sadly Bouncer does not reciprocate this love.
\item[] \myanswer{$\exists x (\atom{M}{x} \eand \atom{L}{x,b} \eand \enot \atom{L}{b,x})$}
\end{compactlist}

\problempart 
Using the following symbolization key:
\begin{ekey}
\item[\text{domain}] all animals
\item[\atom{D}{x}] \gap{x} is a dog
\item[\atom{S}{x}] \gap{x} likes samurai movies
\item[\atom{L}{x,y}] \gap{x} is larger than \gap{y}
\item[r] Rave
\item[h] Shane
\item[d] Daisy
\end{ekey}
symbolize the following sentences in FOL:
\begin{compactlist}
\item Rave is a dog who likes samurai movies.
\item[] \myanswer{$\atom{D}{r} \eand \atom{S}{r}$}
\item Rave, Shane, and Daisy are all dogs.
\item[] \myanswer{$\atom{D}{r} \eand \atom{D}{h} \eand \atom{D}{d}$}
\item Shane is larger than Rave, and Daisy is larger than Shane.
\item[] \myanswer{$\atom{L}{h,r} \eand \atom{L}{d,h}$}
\item All dogs like samurai movies.
\item[] \myanswer{$\forall x(\atom{D}{x} \eif \atom{S}{x})$}
\item Only dogs like samurai movies.
\item[] \myanswer{$\forall x(\atom{S}{x} \eif \atom{D}{x})$\\
Comment: the FOL sentence just written does not require that anyone likes samurai movies. The English sentence might suggest that at least some dogs \emph{do} like samurai movies?}
\item There is a dog that is larger than Shane.
\item[] \myanswer{$\exists x (\atom{D}{x} \eand \atom{L}{x,h})$}
\item If there is a dog larger than Daisy, then there is a dog larger than Shane.
\item[] \myanswer{$\exists x (\atom{D}{x} \eand \atom{L}{x,d}) \eif \exists x(\atom{D}{x} \eand \atom{L}{x,h})$}
\item No animal that likes samurai movies is larger than Shane.
\item[] \myanswer{$\forall x (\atom{S}{x} \eif \enot \atom{L}{x,h})$}
\item No dog is larger than Daisy.
\item[] \myanswer{$\forall x (\atom{D}{x} \eif \enot \atom{L}{x,d})$}
\item Any animal that dislikes samurai movies is larger than Rave.
\item[] \myanswer{$\forall x (\enot \atom{S}{x} \eif \atom{L}{x,r})$\\
Comment: this is very poor, though! For `dislikes' does not mean the same as `does not like'.}
\item There is an animal that is between Rave and Shane in size.
\item[] \myanswer{$\exists x((\atom{L}{r,x} \eand \atom{L}{x,h}) \eor (\atom{L}{h,x} \eand \atom{L}{x,r}))$}
\item There is no dog that is between Rave and Shane in size.
\item[] \myanswer{$\forall x \bigl(\atom{D}{x} \eif \enot\bigl[(\atom{L}{b,x} \eand \atom{L}{x,h}) \eor (\atom{L}{h,x} \eand \atom{L}{x,r})\bigr]\bigr)$}
\item No dog is larger than itself.
\item[] \myanswer{$\forall x(\atom{D}{x} \eif \enot \atom{L}{x,x})$}
\item Every dog is larger than some dog.
\item[] \myanswer{$\forall x (\atom{D}{x} \eif \exists y(\atom{D}{y} \eand \atom{L}{x,y}))$\\
Comment: the English sentence is potentially ambiguous here. I have resolved the ambiguity by assuming it should be paraphrased by `for every dog, there is a dog smaller than it'.}
\item There is an animal that is smaller than every dog.
\item[] \myanswer{$\exists x \forall y(\atom{D}{y} \eif \atom{L}{y,x})$}
\item If there is an animal that is larger than any dog, then that animal does not like samurai movies.
\item[] \myanswer{$\forall x (\forall y (\atom{D}{y} \eif \atom{L}{x,y}) \eif \enot \atom{S}{x})$\\
Comment: I have assumed that `larger than any dog' here means `larger than every dog'.}
\end{compactlist}

\problempart
\label{pr.QLcandies}
Using the symbolization key given, translate each English-language sentence into FOL.
\begin{ekey}
\item[\text{domain}] candies
\item[\atom{C}{x}] \gap{x} has chocolate in it
\item[\atom{M}{x}] \gap{x} has marzipan in it
\item[\atom{S}{x}] \gap{x} has sugar in it
\item[\atom{T}{x}] Boris has tried \gap{x}
\item[\atom{B}{x,y}] \gap{x} is better than \gap{y}
\end{ekey}
\begin{compactlist}
\item Boris has never tried any candy.
\item Marzipan is always made with sugar.
\item Some candy is sugar-free.
\item The very best candy is chocolate.
\item No candy is better than itself.
\item Boris has never tried sugar-free chocolate.
\item Boris has tried marzipan and chocolate, but never together.
%\item Boris has tried nothing that is better than sugar-free marzipan.
\item Any candy with chocolate is better than any candy without it.
\item Any candy with chocolate and marzipan is better than any candy that lacks both.
\end{compactlist}

\problempart
Using the following symbolization key:
\begin{ekey}
\item[\text{domain}] people and dishes at a potluck
\item[\atom{R}{x}] \gap{x} has run out
\item[\atom{T}{x}] \gap{x} is on the table
\item[\atom{F}{x}] \gap{x} is food
\item[\atom{P}{x}] \gap{x} is a person
\item[\atom{L}{x,y}] \gap{x} likes \gap{y}
\item[e] Eli
\item[f] Francesca
\item[g] the guacamole
\end{ekey}
symbolize the following English sentences in FOL:
\begin{compactlist}
\item All the food is on the table.
\item[] \myanswer{$\forall x(\atom{F}{x} \eif \atom{T}{x})$}
\item If the guacamole has not run out, then it is on the table.
\item[] \myanswer{$\enot \atom{R}{g} \eif \atom{T}{g}$}
\item Everyone likes the guacamole.
\item[] \myanswer{$\forall x (\atom{P}{x} \eif \atom{L}{x,g})$}
\item If anyone likes the guacamole, then Eli does.
\item[] \myanswer{$\exists x (\atom{P}{x} \eand \atom{L}{x,g}) \eif \atom{L}{e,g}$}\item Francesca only likes the dishes that have run out.
\item[] \myanswer{$\forall x \bigl[(\atom{L}{f,x} \eand \atom{F}{x}) \eif \atom{R}{x}\bigr]$}
\item Francesca likes no one, and no one likes Francesca.
\item[] \myanswer{$\forall x\bigl[\atom{P}{x} \eif (\enot \atom{L}{f,x} \eand \enot \atom{L}{x,f})\bigr]$}
\item Eli likes anyone who likes the guacamole.
\item[] \myanswer{$\forall x ((\atom{P}{x} \eand \atom{L}{x,g}) \eif \atom{L}{e,x})$}
\item Eli likes anyone who likes the people that he likes.
\item[] \myanswer{$\forall x \bigl[\bigl(\atom{P}{x} \eand \forall y[(\atom{P}{y} \eand \atom{L}{e,y}) \eif \atom{L}{x,y}]\bigr) \eif \atom{L}{e,x}\bigr]$}
\item If there is a person on the table already, then all of the food must have run out.
\item[] \myanswer{$\exists x(\atom{P}{x} \eand \atom{T}{x}) \eif \forall x(\atom{F}{x} \eif \atom{R}{x})$}
\end{compactlist}

\solutions
\problempart
\label{pr.FOLballet}
Using the following symbolization key:
\begin{ekey}
\item[\text{domain}] people
\item[\atom{D}{x}] \gap{x} dances ballet
\item[\atom{F}{x}] \gap{x} is female
\item[\atom{M}{x}] \gap{x} is male
\item[\atom{C}{x,y}] \gap{x} is a child of \gap{y}
\item[\atom{S}{x,y}] \gap{x} is a sibling of \gap{y}
\item[e] Elmer
\item[j] Jane
\item[p] Patrick
\end{ekey}
symbolize the following sentences in FOL:
\begin{compactlist}
\item All of Patrick's children are ballet dancers.
\item[] \myanswer{$\forall x(\atom{C}{x,p} \eif \atom{D}{x})$}
\item Jane is Patrick's daughter.
\item[] \myanswer{$\atom{C}{j,p} \eand \atom{F}{j}$}
\item Patrick has a daughter.
\item[] \myanswer{$\exists x(\atom{C}{x,p} \eand \atom{F}{x})$}
\item Jane is an only child.
\item[] \myanswer{$\enot \exists x \atom{S}{x,j}$}
\item All of Patrick's sons dance ballet.
\item[] \myanswer{$\forall x\bigl[(\atom{C}{x,p} \eand \atom{M}{x}) \eif \atom{D}{x}\bigr]$}
\item Patrick has no sons.
\item[] \myanswer{$\enot \exists x(\atom{C}{x,p} \eand \atom{M}{x})$}
\item Jane is Elmer's niece.
\item[] \myanswer{$\exists x(\atom{S}{x,e} \eand \atom{C}{j,x} \eand \atom{F}{j})$}
\item Patrick is Elmer's brother.
\item[] \myanswer{$\atom{S}{p,e} \eand \atom{M}{p}$}
\item Patrick's brothers have no children.
\item[] \myanswer{$\forall x\bigl[(\atom{S}{p,x} \eand \atom{M}{x}) \eif \enot \exists y\, \atom{C}{y,x}\bigr]$}
\item Jane is an aunt.
\item[] \myanswer{$\atom{F}{j} \eand \exists x(\atom{S}{x,j} \eand \exists y \atom{C}{y,x})$}
\item Everyone who dances ballet has a brother who also dances ballet.
\item[] \myanswer{$\forall x\bigl[\atom{D}{x} \eif \exists y(\atom{M}{y} \eand \atom{S}{y,x} \eand \atom{D}{y})\bigr]$}
\item Every woman who dances ballet is the child of someone who dances ballet.
\item[] \myanswer{$\forall x\bigl[(\atom{F}{x} \eand \atom{D}{x}) \eif \exists y(\atom{C}{x,y} \eand \atom{D}{y})\bigr]$}
\end{compactlist}


\chapter{Identity}\label{sec.identity}\setcounter{ProbPart}{0}
%\problempart
%\label{pr.FOLcandies}
%Using the following symbolization key:
%\begin{ekey}
%\item[\text{domain}] candies
%\item[\atom{C}{x}] \gap{x} has chocolate in it
%\item[\atom{M}{x}] \gap{x} has marzipan in it
%\item[\atom{S}{x}] \gap{x} has sugar in it
%\item[\atom{T}{x}] Boris has tried \gap{x}
%\item[\atom{B}{x,y}] \gap{x} is better than \gap{y}
%\end{ekey}
%symbolize the following English sentences in FOL:\\
%\myanswer{Comment: these are deliberately tricky. What follows is the \emph{best} we can offer in FOL, for each of these sentences. Some are not great.}
%\begin{compactlist}
%\item Boris has never tried any candy.
%\item[] \myanswer{$\forall x(\atom{C}{x} \eif \enot \atom{T}{x})$}
%\item Marzipan is always made with sugar.
%\item[] \myanswer{$\forall x(\atom{M}{x} \eif \atom{S}{x})$}
%\item Some candy is sugar-free.
%\item[] \myanswer{$\exists x \enot \atom{S}{x}$}
%\item The very best candy is chocolate.
%\item[] \myanswer{Simply can't be done! The best we can offer is as in answer to 8.}
%\item No candy is better than itself.
%\item[] \myanswer{$\forall x \enot \atom{B}{x,x}$}
%\item Boris has never tried sugar-free chocolate.
%\item[] \myanswer{$\forall x((\atom{C}{x} \eand \atom{S}{x}) \eif \enot \atom{T}{x})$}
%\item Boris has tried marzipan and chocolate, but never together.
%\item[] \myanswer{$\exists x(\atom{M}{x} \eand \atom{T}{x}) \eand \exists x(\atom{C}{x} \eand \atom{T}{x}) \eand \forall x ((\atom{M}{x} \eand \atom{C}{x}) \eif \enot \atom{T}{x})$}
%%\item Boris has tried nothing that is better than sugar-free marzipan.
%\item Any candy with chocolate is better than any candy without it.
%\item[] \myanswer{$\forall x(\atom{C}{x} \eif \forall (\enot \atom{C}{y} \eif \atom{B}{x,y}))$}
%\item Any candy with chocolate and marzipan is better than any candy that lacks both.
%\item[] \myanswer{$\forall x\bigl[(\atom{C}{x} \eand \atom{M}{x})\eif \forall \bigl((\enot \atom{C}{y} \eand \enot \atom{M}{y}) \eif Bxy\bigr)\bigr]$}
%\end{compactlist}

\problempart Consider the sentence,
\begin{enumerate}[start=19,ref=\arabic*]
	\item\label{except2} Every officer except Pavel owes money to Hikaru.
\end{enumerate}
Symbolize this sentence, using `$\atom{F}{x}$' for `\gap{x} is an officer'.  Are you confident that your symbolization is true if, and only if, \cref*{except2} is true?  What happens if every officer owes money to Hikaru, Pavel does not, but Pavel isn't an officer? \myanswer{Most people, including most linguists thinking about `except', read \cref*{except2} as entailing all three of the following:
\begin{enumerate}
	\item Every officer who is not Pavel owes money to Hikaru.
	\item Pavel does not owe money to Hikaru.
	\item Pavel is an officer.
\end{enumerate}
So it can be symbolized as `$\forall x((\atom{F}{x} \eand \enot x = p) \eif \atom{O}{x,h}) \eand \enot \atom{O}{p,h} \eand \atom{F}{p}$'.}

\problempart Explain why:
	\begin{compactlist}
		\item `$\exists x \forall y(\atom{A}{y} \eiff x= y)$' is a good symbolization of `there is exactly one apple'.
		\item[] \myanswer{We might naturally read this in English thus: 
		\begin{itemize}
			\item There is something, $x$, such that, if you choose any object at all, if you chose an apple then you chose $x$ itself, and if you chose $x$ itself then you chose an apple.
		\end{itemize}
		The $x$ in question must therefore be the one and only thing which is an apple.}
		\item `$\exists x \exists y \bigl[\enot x = y \eand \forall z(\atom{A}{z} \eiff (x= z \eor y = z)\bigr]$' is a good symbolization of `there are exactly two apples'.
		\item[] \myanswer{Similarly to the above, we might naturally read this in English thus: 
		\begin{itemize}
			\item There are two distinct things, $x$ and $y$, such that if you choose any object at all, if you chose an apple then you either chose $x$ or $y$, and if you chose either $x$ or $y$ then you chose an apple.
		\end{itemize}
		The $x$ and $y$ in question must therefore be the only things which are apples, and since they are distinct, there are two of them.}
	\end{compactlist}


\chapter{Sentences of FOL}\setcounter{ProbPart}{0}
\problempart
\label{pr.freeFOL}
Identify which variables are bound and which are free.
\myanswer{We underline the bound variables, and overline the free variables.}
\begin{compactlist}
\item $\exists x\, \atom{L}{\underline{x},\overline{y}} \eand \forall y\,\atom{L}{\underline{y},\overline{x}}$
\item $\forall x\, \atom{A}{\underline{x}} \eand \atom{B}{\overline{x}}$
\item $\forall x (\atom{A}{\underline{x}} \eand \atom{B}{\underline{x}}) \eand \forall y(\atom{C}{\overline{x}} \eand \atom{D}{\underline{y}})$
\item $\forall x\exists y[\atom{R}{\underline{x},\underline{y}} \eif (\atom{J}{\overline{z}} \eand \atom{K}{\underline{x}})] \eor \atom{R}{\overline{y},\overline{x}}$
\item $\forall x_1(\atom{M}{\overline{x_2}} \eiff \atom{L}{\overline{x_2},\underline{x_1}}) \eand \exists x_2\, \atom{L}{\overline{x_3},\underline{x_2}}$
\end{compactlist}


\chapter{Definite descriptions}\setcounter{ProbPart}{0}
\problempart
Using the following symbolization key:
\begin{ekey}
\item[\text{domain}] people
\item[\atom{K}{x}] \gap{x} knows the combination to the safe
\item[\atom{S}{x}] \gap{x} is a spy
\item[\atom{V}{x}] \gap{x} is a vegetarian
\item[\atom{T}{x,y}] \gap{x} trusts \gap{y}
\item[h] Hofthor
\item[i] Ingmar
\end{ekey}
symbolize the following sentences in FOL:
\begin{compactlist}
\item Hofthor trusts a vegetarian.
\item[] \myanswer{$\exists x(\atom{V}{x} \eand \atom{T}{h,x})$}
\item Everyone who trusts Ingmar trusts a vegetarian.
\item[] \myanswer{$\forall x\bigl[\atom{T}{x,i} \eif \exists y(\atom{T}{x,y} \eand \atom{V}{y})\bigr]$}
\item Everyone who trusts Ingmar trusts someone who trusts a vegetarian.
\item[] \myanswer{$\forall x\bigl[\atom{T}{x,i} \eif \exists y\bigr(\atom{T}{x,y} \eand \exists z(\atom{T}{y,z} \eand \atom{V}{z})\bigr)\bigr]$}
\item Only Ingmar knows the combination to the safe.
\item[] \myanswer{$\forall x(\atom{K}{x} \eif x = i)$\\Comment: does the English claim entail that Ingmar \emph{does} know the combination to the safe? If so, then we should formalize this with a `$\eiff$'.}
\item Ingmar trusts Hofthor, but no one else.
\item[] \myanswer{$\forall x(\atom{T}{i,x} \eiff x = h)$}
\item The person who knows the combination to the safe is a vegetarian.
\item[] \myanswer{$\exists x\bigl[\atom{K}{x} \eand \forall y(\atom{K}{y} \eif x = y) \eand \atom{V}{x}\bigr]$}
\item The person who knows the combination to the safe is not a spy.
\item[] \myanswer{$\exists x\bigl[\atom{K}{x} \eand \forall y(\atom{K}{y} \eif x = y) \eand \enot \atom{S}{x}\bigr]$\\
Comment: the scope of negation is potentially ambiguous here; I have read it as \emph{inner} negation.}
\end{compactlist}


\solutions
\problempart
\label{pr.FOLcards}
Using the following symbolization key:
\begin{ekey}
\item[\text{domain}] cards in a standard deck
\item[\atom{B}{x}] \gap{x} is black
\item[\atom{C}{x}] \gap{x} is a club
\item[\atom{D}{x}] \gap{x} is a deuce
\item[\atom{J}{x}] \gap{x} is a jack
\item[\atom{M}{x}] \gap{x} is a man with an axe
\item[\atom{O}{x}] \gap{x} is one-eyed
\item[\atom{W}{x}] \gap{x} is wild
\end{ekey}
symbolize each sentence in FOL:
\begin{compactlist}
\item All clubs are black cards.
\item[] \myanswer{$\forall x (\atom{C}{x} \eif \atom{B}{x})$}
\item There are no wild cards.
\item[] \myanswer{$\enot \exists x\, \atom{W}{x}$}
\item There are at least two clubs.
\item[] \myanswer{$\exists x \exists y(\enot x = y \eand \atom{C}{x} \eand \atom{C}{y})$}
\item There is more than one one-eyed jack.
\item[] \myanswer{$\exists x \exists y(\enot x = y \eand \atom{J}{x} \eand \atom{O}{x}  \eand \atom{J}{y} \eand \atom{O}{y})$}
\item There are at most two one-eyed jacks.
\item[] \myanswer{$\forall x \forall y \forall z\bigl[(\atom{J}{x} \eand \atom{O}{x} \eand \atom{J}{y} \eand \atom{O}{y} \eand \atom{J}{z} \eand \atom{O}{z}) \eif (x = y \eor x = z \eor y = z)\bigr]$}
\item There are two black jacks.
\item[] \myanswer{$\exists x \exists y(\enot x = y \eand \atom{B}{x} \eand \atom{J}{x} \eand \atom{B}{y} \eand \atom{J}{y})$\\
Comment: I am reading this as `there are \emph{at least} two\ldots'. If the suggestion was that there are \emph{exactly} two, then a different FOL sentence would be required, namely:\\
$\exists x \exists y \bigl(\enot x = y \eand \atom{B}{x} \eand \atom{J}{x} \eand \atom{B}{y} \eand \atom{J}{y} \eand \forall z[(\atom{B}{z} \eand \atom{J}{z}) \eif (x = z \eor y = z)]\bigr)$}
\item There are four deuces.
\item[] \myanswer{$\exists w \exists x \exists y \exists z(\enot w = x \eand \enot w = y \eand \enot w = z \eand \enot x = y \eand \enot x = z \eand \enot y = z \eand \atom{D}{w} \eand \atom{D}{x} \eand \atom{D}{y} \eand \atom{D}{z})$\\
Comment: I am reading this as `there are \emph{at least} four\ldots'. If the suggestion is that there are \emph{exactly} four, then we should offer instead:\\
$\exists w \exists x \exists y \exists z\bigl(\enot w = x \eand \enot w = y \eand \enot w = z \eand \enot x = y \eand \enot x = z \eand \enot y = z \eand \atom{D}{w} \eand \atom{D}{x} \eand \atom{D}{y} \eand \atom{D}{z} \eand \forall v[\atom{D}{v} \eif (v = w \eor v = x \eor v = y \eor v =z)]\bigr)$}
\item The deuce of clubs is a black card.
\item[] \myanswer{$\exists x \bigl[\atom{D}{x} \eand \atom{C}{x} \eand \forall y\bigl((\atom{D}{y} \eand \atom{C}{y}) \eif x = y\bigr) \eand \atom{B}{x}\bigr]$}
\item One-eyed jacks and the man with the axe are wild.
\item[] \myanswer{$\forall x \bigl[(\atom{J}{x} \eand \atom{O}{x}) \eif \atom{W}{x}\bigr] \eand \exists x\bigl[\atom{M}{x} \eand \forall y(\atom{M}{y} \eif x = y) \eand \atom{W}{x}\bigr]$}
\item If the deuce of clubs is wild, then there is exactly one wild card.
\item[] \myanswer{$\exists x \bigl(\atom{D}{x} \eand \atom{C}{x} \eand \forall y \bigl[(\atom{D}{y} \eand \atom{C}{y}) \eif x= y\bigr] \eand \atom{W}{x}\bigr) \eif \exists x \bigl(\atom{W}{x} \eand \forall y(\atom{W}{y} \eif x = y)\bigr)$\\
Comment: if there is not exactly one deuce of clubs, then the above sentence is true. Maybe that's the wrong verdict. Perhaps the sentence should definitely be taken to imply that there is one and only one deuce of clubs, and then express a conditional about wildness. If so, then we might symbolize it thus:
\\$\exists x \bigl(\atom{D}{x} \eand \atom{C}{x} \eand \forall y \bigl[(\atom{D}{y} \eand \atom{C}{y}) \eif x = y\bigr] \eand \bigl[\atom{W}{x} \eif \forall y (\atom{W}{y} \eif x = y)\bigr]\bigl)$}
\item The man with the axe is not a jack.
\item[] \myanswer{$\exists x \bigl[\atom{M}{x} \eand \forall y(\atom{M}{y} \eif x = y) \eand \enot \atom{J}{x}\bigr]$}
\item The deuce of clubs is not the man with the axe.
\item[] \myanswer{$\exists x \exists y\bigl(\atom{D}{x} \eand \atom{C}{x} \eand \forall z[(\atom{D}{z} \eand \atom{C}{z}) \eif x = z] \eand \atom{M}{y} \eand \forall z(\atom{M}{z} \eif y = z) \eand \enot x = y\bigr)$}

\end{compactlist}

\

\problempart Using the following symbolization key:
\begin{ekey}
\item[\text{domain}] animals in the world
\item[\atom{B}{x}] \gap{x} is in Farmer Brown's field
\item[\atom{H}{x}] \gap{x} is a horse
\item[\atom{P}{x}] \gap{x} is a Pegasus
\item[\atom{W}{x}] \gap{x} has wings
\end{ekey}
symbolize the following sentences in FOL:
\begin{compactlist}
\item There are at least three horses in the world.
\item[] \myanswer{$\exists x \exists y \exists z (\enot x = y \eand \enot x = z \eand \enot y = z \eand \atom{H}{x} \eand \atom{H}{y} \eand \atom{H}{z})$}
\item There are at least three animals in the world.
\item[] \myanswer{$\exists x \exists y \exists z (\enot x = y \eand \enot x = z \eand \enot y = z)$}
\item There is more than one horse in Farmer Brown's field.
\item[] \myanswer{$\exists x \exists y (\enot x = y \eand \atom{H}{x} \eand \atom{H}{y} \eand \atom{B}{x} \eand \atom{B}{y})$}
\item There are three horses in Farmer Brown's field.
\item[] \myanswer{$\exists x \exists y \exists z(\enot x = y \eand \enot x = z \eand \enot y = z \eand \atom{H}{x} \eand \atom{H}{y} \eand \atom{H}{z} \eand \atom{B}{x} \eand \atom{B}{y} \eand \atom{B}{z})$\\Comment: I have read this as `there are \emph{at least} three\ldots'. If the suggestion was that there are \emph{exactly} three, then a different FOL sentence would be required.}
\item There is a single winged creature in Farmer Brown's field; any other creatures in the field must be wingless.
\item[] \myanswer{$\exists x\bigl[\atom{W}{x} \eand \atom{B}{x} \eand \forall y\bigl((\atom{W}{y} \eand \atom{B}{y}) \eif x = y)\bigr]$}
\item The Pegasus is a winged horse.
\item[] \myanswer{$\exists x \bigl[\atom{P}{x} \eand \forall y(\atom{P}{y} \eif x = y) \eand \atom{W}{x} \eand \atom{H}{x}\bigr]$}
\item The animal in Farmer Brown's field is not a horse.
\item[] \myanswer{$\exists x \bigl[ Bx \eand \forall y (\atom{B}{y} \eif x = y) \eand \enot \atom{H}{x}\bigr]$\\Comment: the scope of negation might be ambiguous here; I have read it as \emph{inner} negation.}
\item The horse in Farmer Brown's field does not have wings.
\item[] \myanswer{$\exists x \bigl[\atom{H}{x} \eand \atom{B}{x} \eand \forall y \bigl((\atom{H}{y} \eand \atom{B}{y}) \eif x = y\bigr) \eand \enot \atom{W}{x}\bigr]$\\Comment: the scope of negation might be ambiguous here; I have read it as \emph{inner} negation.}

\end{compactlist}

\problempart
In this chapter, we symbolized `Nick is the traitor' by `$\exists x (\atom{T}{x} \eand \forall y(\atom{T}{y} \eif x = y) \eand x = n)$'. Explain why these would be equally good symbolisations:
	\begin{itemize}
		\item $\atom{T}{n} \eand \forall y(\atom{T}{y} \eif n = y)$
		\item[] \myanswer{This sentence requires that Nick is a traitor, and that Nick alone is a traitor. Otherwise put, there is one and only one traitor, namely, Nick. Otherwise put: Nick is the traitor.}
		\item $\forall y(\atom{T}{y} \eiff y = n)$
		\item[] \myanswer{This sentence can be understood thus: Take anything you like; now, if you chose a traitor, you chose Nick, and if you chose Nick, you chose a traitor. So there is one and only one traitor, namely, Nick, as required.}
	\end{itemize}

\chapter{Ambiguity}
\setcounter{ProbPart}{0}

\practiceproblems
\problempart
Each of the following sentences is ambiguous. Provide a symbolization key for each, and symbolize all readings.
\begin{compactlist}
	\item No one likes a quitter.
	\myanswer{\begin{ekey}
		\item[\text{domain}] people
		\item[\atom{Q}{x}] \gap{x} is a quitter
		\item[\atom{L}{x,y}] \gap{x} likes \gap{y}
	\end{ekey}}
	\item[] \myanswer{$\forall x(\atom{Q}{x} \eif \forall y\,\enot\atom{L}{y,x})$
	(all quitters are disliked by everyone)\\
	$\exists x(\atom{Q}{x} \eand \forall y\,\enot\atom{L}{y,x})$ (A
	specific quitter is disliked by everyone)}
	\item CSI found only red hair at the scene.
	\myanswer{\begin{ekey}
		\item[\text{domain}] items of evidence
		\item[\atom{R}{x}] \gap{x} is red
		\item[\atom{H}{x}] \gap{x} is hair
		\item[\atom{F}{x,y}] \gap{x} found \gap{y} (at the scene)
		\item[c] CSI
	\end{ekey}}
	\item[] \myanswer{$\forall x(\atom{F}{c,x} \eif (\atom{R}{x} \eand \atom{H}{x}))$
	(only hair was found, and it's all red)\\
	$\forall x((\atom{F}{c,x}  \eand \atom{H}{x}) \eif \atom{R}{x})$
	(only \emph{red} hair was found, but possibly other things that
	aren't hair)}
	\item Smith's murderer hasn't been arrested.
	\myanswer{\begin{ekey}
		\item[\text{domain}] people
		\item[\atom{A}{x}] \gap{x} has been arrested
		\item[\atom{M}{x,y}] \gap{x} murdered \gap{y}
		\item[m] Smith
	\end{ekey}}
	\item[] \myanswer{$\exists x(\forall y(\atom{M}{y, m} \eiff y=x)
	\eand \enot\atom{A}{x})$ (Smith's murderer has: not been arrested)\\
	$\enot\exists x(\forall y(\atom{M}{y, m} \eiff y=x)
	\eand \atom{A}{x})$ (It's not the case that Smith's murderer has
	been arrested; in fact there may not be a ``Smith's murderer''.)}
\end{compactlist}
