%!TEX root = forallxyyc.tex

\chapter{Notes on accessibility}

Special attention has been paid to make this HTML version as
accessible as possible, especially to readers using Assistive
Technology (AT), such as screen readers like
\href{https://www.nvaccess.org/download/}{NVDA}
and
\href{https://www.freedomscientific.com/products/software/jaws/}{JAWS}.
Screen readers are software designed to enable users with low or no
vision to access and navigate web pages like this book. Some sighted
users also sometimes prefer text-to-speech (TTS) software to read web
pages aloud. TTS software has fewer features than screen readers
designed for non-sighted users and may not convert everything in this
book to audio.

If you are using the HTML version of this book with a screen reader or
Braille terminal, or are helping a student who relies on such tools
(e.g., as instructor or accessibility advisor), and you have feedback
please \href{mailto:rzach@ucalgary.ca}{get in touch}!

\paragraph{Navigation and styles} We have adopted the HTML style
provided by \href{https://vlmantova.github.io/bookml/}{BookML},
developed for the Leeds mathematics department by
\href{https://eps.leeds.ac.uk/maths/staff/4058/dr-vincenzo-l-mantova}{Vincenzo
Mantova}. It provides a collapsible navigation menu, buttons at the
top of each page to select the font size, switch between serif and
sans-serif font, and black-on-white, dark-on-sepia, or light-on-dark
display options.

The body of the page starts with an invisible link to skip to the
beginning of the content, past the navigation sidebar and style
buttons. The buttons are, in order: 
\begin{itemize}
  \item ``toggle sidebar'' to turn the navigation bar on or off;
  \item ``font settings'' for font size, serif or sans serif font, and
    light, sepia, or dark modes; 
  \item ``download'' for links to download the PDF versions; and 
  \item ``information about the toolbar'' for a popup that shows
    keyboard shortcuts for navigation.
\end{itemize}
Those keyboard shortcuts are: left and right arrow for moving to the
previous and the next page, and `s' to toggle the navigation sidebar.

\paragraph{Symbols and formulas} 

Logical symbols and formulas in this book are converted to
\href{https://www.w3.org/Math/}{MathML} and displayed using
\href{https://www.mathjax.org/}{MathJax}. MathJax provides additional
\href{https://docs.mathjax.org/en/latest/basic/accessibility.html}{accessibility
features} for formulas, which are found in the MathJax context menu.
With a mouse, you can right click on any formula to activate it.
Screen readers will announce that formulas can be activated (e.g.,
NVDA says ``application clickable'' before every formula to indicate
the presence of the MathJax menu.) If your screen reader does not read
out formulas such as $\forall x(A(x) \land B(x))$ you may need to
activate Speech Output in the Speech submenu of the MathJax
Accessibility menu and reload the page.

Here are some of the more common symbols. Different screen readers and
TTS applications will pronouce them differently, and sometimes not at
all. After each symbol, we provide the customary pronounciation used
by logicians.
\begin{itemize}
  \item Uppercase and lowercase italic letters like $A$ (``upper A'')
  and $x$ (``lower x'')
  \item Uppercase and lowercase script letters like $\metav{A}$
  (``script upper A'') and $\metav{x}$ (``script lower x'')
  \item Logical connectives: $\enot$ (``not''), $\eor$ (``or''),
  $\eand$ (``and''), $\eif$ (``only if''), $\eiff$ (``if and only if''), $\ered$ (``contradiction'')
  \item Quantifiers: $\forall$ (``for all''), $\exists$ (``there exists'')
  \item Metatheoretical symbols: $\therefore$ (``therefore''),
  $\proves$ (``proves''), $\nproves$ (``does not prove''), $\entails$ (``entails''), $\nentails$ (``does not entail'')
  \item Modal operators: $\ebox$ (``box''), $\ediamond$ (``diamond'')
\end{itemize}

Subscripts should be pronounced by screen readers, although
they may not indicate that they are subscripts. The formula $A_2$ may
be read out as ``upper A sub 2'' or just as ``A2''.

The expressions `\blank{}' and `\ifeff' (`if' with two `f's) are used
throughout the textbook. In the HTML version, `\blank{}' includes an
invisible spelled-out word `blank' which screen readers should read
out loud. The abbreviation `\ifeff' is short for `if and only if'. In
the HTML version it is coded using an invisible zero-space word joiner
character which should make screen readers pronounce `\ifeff' as
`if-eff' or `aye-eff-eff'.

\paragraph{Proofs} The natural deduction proofs in
\cref{ch.NDTFL,ch.NDFOL,ch.ML} use vertical lines to indicate where
subproofs start and end. Such vertical lines extend from the
assumption line of the subproof to its last line and are displayed
between the line numbers and the formulas in any given line. This
makes proofs a special challenge for users with low vision or complete
loss of vision.

To make these proofs accessible in this HTML version, proofs are coded
as tables. Each table line has four columns: the line number, a
subproof level indicator, a formula, and a justification. The subproof
level indicator is a number recording how many nested subproofs the
current line is contained in. It is 0 if the line is not contained in
a subproof, 1 if it is in a subproof, 2 if it is in a subproof nested
within another subproof, and so on. When reading out a subproof level
indicator, screen readers should also announce if a subproof has just
been closed on the previous line, and when a new subproof starts. The
table header rows and subproof level indicators are hidden so that
proofs visually appear as in the printed text.

Here is an example of such a proof:
\begin{fitchproof}
\hypo{wxyz}{(W \eor X) \eor (Y \eor Z)}\PR
\hypo{xy}{X \eif Y}\PR
\hypo{nz}{\enot Z}\PR
\open
	\hypo{wx}{W \eor X}\AS
	\open
		\hypo{w}{W}\AS
		\have{wy1}{W \eor Y}\oi{w}
	\close
	\open
		\hypo{x}{X}\AS
		\have{y1}{Y}\ce{xy, x}
		\have{wy2}{W \eor Y}\oi{y1}
	\close
	\have{wy3}{W \eor Y}\oe{wx, w-wy1, x-wy2}
\close
\open
	\hypo{yz}{Y \eor Z}\AS
	\have{y}{Y}\ds{yz, nz}
	\have{wy}{W \eor Y}\oi{y}
\close
\have{wy4}{W \eor Y}\oe{wxyz, wx-wy3, yz-wy}
\end{fitchproof}
It has 14 lines, with 3 premises, 2 levels of subproof nesting, and
two pairs of adjacent subproofs. For instance, the subproof beginning
on line 5 ends at line 6, and line 7 starts another subproof. So the
subproof levels of lines 6 and 7 is the same, but lines 6 and 7 are in
different subproofs. If you cannot see the subproof lines, you have to
pay special attention to how the subproof level numbers change
\emph{and} when a formula is an assumption. A screen reader should
read line 7 approximately as: ``7, close subproof, 2, open subproof,
upper X, AS.'' The abbreviations `\PR' (for `Premise') and `\AS' (for
`Assumption') are coded in the HTML version using the `\verb|abbr|'
tag.

Note that simple text-to-speech applications may not read out complex
tables such as the ones containing proofs or truth tables at all.

\paragraph{Diagrams}

The text also contains a few diagrams. In the HTML version, these are
provided with image descriptions, e.g.,
\begin{center}
\begin{tikzpicture}
	\node[circle, grey_shape] (cat1) {A};
	\node[right=10pt of cat1, diamond, phantom_shape] (cat2)  { } ;
	\node[right=10pt of cat2, circle, white_shape] (cat3)  {C} ;
	\node[right=10pt of cat3, white_shape] (cat4)  {D};
\end{tikzpicture}
\bmlDescription{There are three shapes: the first shape is a grey
circle labelled A, some empty space, the second shape is a white
circle labelled C, and the fourth a white square labelled D.}
\end{center}
The image description here should read: ``There are three shapes: the
first shape is a grey circle labelled A, some empty space, the second
shape is a white circle labelled C, and the fourth a white square
labelled~D.''

\paragraph{Table of symbols}

What follows is a table of all the symbols used in the text, how they
are most likely to be pronounced by screen readers, and what they are
called in the text. In the rightmost column we provide a suggested way
to enter them using ASCII characters, if inserting special symbols (in
a homework assignment or email to your instructor, say) is not an
option.

\begin{tabular}{lllll}
  \lxBeginTableHead{}Symbol\lxTableColumnHead{} & Pronounciation\lxTableColumnHead{} & Meaning\lxTableColumnHead{} & ASCII equivalent\lxTableColumnHead{}\\
  \hline\lxEndTableHead
  $\enot$ & not sign & logical not & \verb+~+ or \verb+-+\\
  $\eor$ & or & logical or & \verb+\/+\\
  $\eand$ & and & logical and & \verb+/\+ or \verb+&+\\
  $\eif$ & right arrow & conditional & \verb+->+ or \verb+>+\\
  $\eiff$ & left right arrow & biconditional & \verb+<->+ or \verb+<>+\\ 
  $\ered$ & up tack & contradiction & \verb+_|_+ or \verb+!?+\\
  $\forall$ & for all & universal quantifier & \verb|A| or \verb|@|\\
  $\exists$ & there exists & existential quantifier & \verb|E| or \verb|3|\\
  $\therefore$ & therefore & therefore & \verb|:.|\\
  $\proves$ & right tack & proves & \verb+|-+\\
  $\nproves$ & does not prove & does not prove & \verb+|/-+\\
  $\entails$ & true & entails & \verb+|=+\\
  $\nentails$ & not true & does not entail & \verb+|/=+\\
  $\ebox$ & white square & necessary & \verb+[]+\\
  $\ediamond$ & white diamond & possible & \verb+<>+
\end{tabular}

Subscripts can be represented by an underline, e.g., $A_2$ as
\verb|A_2|.
